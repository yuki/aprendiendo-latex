\documentclass[12pt,a4paper,openany]{book}
\usepackage[left=2cm,right=2cm,top=2cm,bottom=2cm]{geometry}
\usepackage[utf8]{inputenc}
\usepackage[spanish]{babel}
%\usepackage{amsmath}
%\usepackage{amsfonts}
%\usepackage{amssymb}
\usepackage[hidelinks]{hyperref} %links in TOC
\usepackage{kpfonts}

% for Systemctl's "box" unicode symbols
\usepackage{pmboxdraw}

% for Systemctl's bullet 25CF unicode symbol
\usepackage{pifont} % for the black circle
\DeclareUnicodeCharacter{25CF}{\resizebox{0.5em}{!}{\ding{108}}}


% color boxes
\usepackage{tcolorbox}
%\newtcolorbox[blend into=figures]{myfigure}[2][]{float=htb,title={#2},#1}

%for source code. Python's Pygments package needed and "-shel-escape" into cli/Texmaker
\tcbuselibrary{minted} %minted is integrated into tcolorbox
\tcbset{listing engine=minted}
%\usepackage{minted}
\usemintedstyle{solarized-dark}


\definecolor{solarizeddark}{HTML}{002b36}
%\definecolor{solarizeddark}{HTML}{f0f0f0}

%for images
\usepackage{graphicx}
\usepackage{wrapfig}


\RequirePackage{blindtext} %for lorem ipsum


\author{Ruben Gomez}
\title{Titulo}



\begin{document}
\maketitle

\tableofcontents

\part{ad}

\chapter{Código fuente}
Ejemplos de código fuente en distintos lenguajes usando \textbf{minted}:

\begin{minted}[bgcolor=solarizeddark]{console}
NAME        MAJ:MIN RM   SIZE RO TYPE MOUNTPOINTS
sda           8:0    0   1,8T  0 disk 
└─sda1        8:1    0   1,8T  0 part /home/backup
sdb           8:16   0   3,6T  0 disk 
└─sdb1        8:17   0   3,6T  0 part /home/disco4tb
sdc           8:32   0 447,1G  0 disk 
├─sdc1        8:33   0   529M  0 part 
├─sdc2        8:34   0   100M  0 part 
├─sdc3        8:35   0    16M  0 part 
└─sdc4        8:36   0 446,5G  0 part 
sdd           8:48   0   3,6T  0 disk 
└─sdd1        8:49   0   3,6T  0 part /home/disco4tb3
sde           8:64   0   3,6T  0 disk 
└─sde1        8:65   0   3,6T  0 part /home/disco4tb2
sdf           8:80   1   7,6G  0 disk 
├─sdf1        8:81   1   200M  0 part 
└─sdf2        8:82   1   7,3G  0 part /media/yuki/Booter
nvme0n1     259:0    0 931,5G  0 disk 
├─nvme0n1p1 259:1    0   512M  0 part /boot/efi
├─nvme0n1p2 259:2    0  29,8G  0 part /
├─nvme0n1p3 259:3    0 745,1G  0 part /home
└─nvme0n1p4 259:4    0 156,2G  0 part 

\end{minted}

Ejemplo de systemctl
%see escaped information inside "||" using color green
\begin{minted}[bgcolor=solarizeddark,escapeinside=||]{console}

|\$| systemctl status gdm3 
|\color{green}{●}| gdm.service - GNOME Display Manager
     Loaded: loaded (/lib/systemd/system/gdm.service; static)
     Active: |\color{green}{active (running)}| since Sat 2021-11-13 21:33:36 CET; 6 days ago
   Main PID: 1052 (gdm3)
      Tasks: 3 (limit: 18937)
     Memory: 4.6M
        CPU: 6.013s
     CGroup: /system.slice/gdm.service
             └─1052 /usr/sbin/gdm3

\end{minted}

Ejemplo de consola: 	
\begin{minted}[bgcolor=solarizeddark]{shell-session}
 sudo systemctl stop mysql
\end{minted}

Ejemplo de SQL, y con inline de mysql como prompt \mintinline[bgcolor=solarizeddark]{console}{ mysql> }

\begin{tcolorbox}[arc=3mm,colback=solarizeddark,title=My heading line]
	\begin{minted}{mysql}
CREATE TABLE Persons (
    PersonID int,
    LastName varchar(255),
    FirstName varchar(255),
    Address varchar(255),
    City varchar(255)
);
	\end{minted}
\end{tcolorbox}


%Ejemplo de listings
%%\lstset{backgroundcolor=\color{gray}, frame=tlb}
%\begin{lstlisting}[language={C}]
%int main(){
%  return 0;
%}
%\end{lstlisting}
%
%%Ejemplo de listings 2
%
%\begin{tcolorbox}[title=My heading line]
%	\begin{lstlisting}[language={SQL}]
%	CREATE TABLE Persons (
%	    PersonID int,
%	    LastName varchar(255),
%	    FirstName varchar(255),
%	    Address varchar(255),
%	    City varchar(255)
%	);
%	\end{lstlisting}
%\end{tcolorbox}



\section{Imágenes}
\begin{wrapfigure}{r}{0.15\textwidth} %this figure will be at the right
    \centering %para centrar la imagen en el hueco
    \vspace{-10pt} %para alinear la imagen con la parte alta del texto
    \includegraphics[width=0.15\textwidth]{img/openlogo-nd.png}
\end{wrapfigure}\Blindtext[1]


\begin{wrapfigure}{l}{0.15\textwidth} %this figure will be at the left
    \centering
    \includegraphics[width=0.15\textwidth]{img/openlogo-nd.png}
\end{wrapfigure}\Blindtext[1]



\begin{figure}
	\centering
	\includegraphics[width=0.5\textwidth]{img/openlogo-nd.png}
	\caption{Ejemplo de figura centrada}
	\label{fig:logo}
\end{figure}

\begin{tcolorbox}{Some title for these equations}
\[ \sin^2x+\cos^2x=1\]

\[ 1 + \frac{1}{\tan^2x}=\frac{1}{\sin^2x}\]
\end{tcolorbox}

%\begin{myfigure}{Some title for these equations}
%\[ \sin^2x+\cos^2x=1\]
%
%\[ 1 + \frac{1}{\tan^2x}=\frac{1}{\sin^2x}\]
%\end{myfigure}

\begin{tcolorbox}
	\centering
	\includegraphics[width=0.5\textwidth]{img/openlogo-nd.png}
\end{tcolorbox}



Ahora vamos a hacer una referencia a la figura \ref{fig:logo} y ver cómo queda y si hago referencia a \pageref{fig:logo} qué sale?

%\includegraphics[width=\textwidth]{img/openlogo-nd.png}	

\Blindtext[1]


\subsection{sub1}
\Blindtext[3]
\subsubsection{sub sub 1}
\Blindtext[3]
\paragraph{parag 1}
\Blindtext[3]
\subparagraph{sub par1}
\Blindtext[3]

	





\chapter{}
\Blindtext[3]
\section{secci 1}
\Blindtext[3]
\subsection{sub1}
\Blindtext[3]
\subsubsection{sub sub 1}




\end{document}


